\documentclass[10pt]{article}

\usepackage[tmargin=0.1in, bmargin=0.25in, lmargin = 0.1in, rmargin = 0.25in]{geometry}
\usepackage{amsmath, amssymb, amsthm}
\usepackage{enumitem}
\usepackage{multicol}
\usepackage{listings}
\usepackage{pstricks}
\usepackage{tikz}
\newcommand*\circled[1]{\tikz[baseline=(char.base)]{
    \node[shape=circle,draw,inner sep=0.5pt] (char) {#1};}}

\makeatletter
\renewcommand*\env@matrix[1][*\c@MaxMatrixCols c]{%
    \hskip -\arraycolsep
    \let\@ifnextchar\new@ifnextchar
    \array{#1}}
\makeatother

\begin{document}
\footnotesize
\begin{multicols}{2}
    \begin{minipage}{0.8\columnwidth}
        \vspace*{-70em}
        \section*{UNIX File System}
        \begin{itemize}
            \setlength\itemsep{0em}
            \item Boot block: contains bootstrap code to boot OS
            \item Super block: describes state of file system size, how many files, location of free space
            \item Inode list: list of inodes
            \item Data blocks: contains file data
        \end{itemize}
        \setlength\itemsep{-0.3em}
        \subsubsection*{Inodes consist of}
        \begin{itemize}
            \setlength\itemsep{0em}
            \item File owner
            \item File type
            \item Access permissions (rw-r--r--)
            \item Size / \# of blocks
            \item Access, Modify, and Status Time Change
            \item Access to data blocks (n data block pointers)
        \end{itemize}
        \vspace*{-2em}
        \subsection*{stat(2)}
        Obtains information about the file pointed to by \textit{path}
        (or in the case of fstat(2), fd).
        If \textit{path} is a symlink, returns info about the link itself.
        \begin{lstlisting}
struct stat {
    mode_t st_mode; /* file type & mode (permissions) */
    ino_t st_ino; /* i-node number (serial number) */
    dev_t st_dev; /* device number (file system) */
    dev_t st_rdev; /* device number for special files */
    nlink_t st_nlink; /* number of links */
    uid_t st_uid; /* user ID of owner */
    gid_t st_gid; /* group ID of owner */
    off_t st_size; /* size in bytes, for regular files */
    struct timespec st_atim; /* last access time */
    struct timespec st_mtim; /* last modification time */
    struct timespec st_ctim; /* file status change time*/
    blksize_t st_blksize; /* best I/O block size */
    blkcnt_t st_blocks; /* # disk blocks allocated */
}

#include <sys/stat.h>
int stat(const char *path, struct stat *sb);
int lstat(const char *path, struct stat *sb);
int fstat(int fd, struct stat *sb);
        \end{lstlisting}
        \paragraph*{st\_mode}
        \begin{description}
            \item[regular] - most comon, interpretation is up to app
            \item[directory] - contains names of other files and pointer to said info
            \item[char special] - used for certain types of devices (eg terminal)
            \item[block special] - used for disk devices
            \item[FIFO] - used for interprocess communication
            \item[socket] - used for network and non-network communication
            \item[symbolic link] - points to another file
        \end{description}
        \subsubsection*{st\_uid, st\_guid}
        Every process has six or more IDs associated with it.
        st\_uid and st\_gid always specify the user owner and group owner of a file
        \begin{tabular}{ll}
            \hline
            real user/group ID                        & who we really are          \\
            effective user/group ID / extra group IDs & used for file access perms \\
            saved set-user-ID / saved set-group-ID    & saved by exec funcs        \\
        \end{tabular}
        \subsubsection*{Access Tests Performed by Kernel:}
        \begin{enumerate}
            \item e-uid == 0 (root), access granted
            \item e-uid == st\_uid, access granted if user permission set
            \item e-gid == st\_gid, access granted if group permission set
            \item other permission bit set, access granted
            \item access denied
        \end{enumerate}
    \end{minipage}
    \begin{minipage}{\columnwidth}
        \vspace*{3em}
        \begin{flushright}
            \subsection*{chmod(2), lchmod(2), fchmod(2)}
            Changes the permission bits on the file.
            \footnotesize
            \begin{lstlisting}
#include <sys/stat.h>
#include <fcntl.h>
int chmod(const char *path, mode_t mode);
int lchmod(const char *path, mode_t mode);
int fchmod(const char *path, mode_t mode);
int fchmodat(int fd, const char *path, mode_t mode, int flag);
returns 0 if OK, -1 on error
            \end{lstlisting}
            \vspace*{-1em}
            \section*{Standard I/O}
            \begin{multicols*}{2}
                \begin{itemize}
                    \item 3 streams automatically created
                          \begin{itemize}
                              \item STDIN\_FILENO
                              \item STDOUT\_FILENO
                              \item STDERR\_FILENO
                          \end{itemize}
                          \columnbreak
                    \item These three are referenced as \textbf{stdin}, \textbf{stdout}
                          , and \textbf{stderr}.
                    \item Defined in stdio.h
                \end{itemize}
            \end{multicols*}
            \vspace*{-5em}
            \subsection*{Streams and FILEs}
            \begin{multicols*}{2}
                \begin{itemize}
                    \item fopen() returns pointed to FILE object
                          \begin{itemize}
                              \item File descripter
                              \item Pointer to stream buffer
                              \item Buffer size
                              \item Number of chars in buffer
                              \item Error flag
                          \end{itemize}
                          \columnbreak
                    \item Applications need not examine FILE object
                    \item Pass FILE pointer as argument to standard I/O funcs
                    \item FILE * is called file pointer
                \end{itemize}
            \end{multicols*}
            \subsubsection*{Six ways to open standard I/O stream}
            \begin{tabular}{|c|c|c|c|c|c|c|}
                \hline
                Restriction                 & r         & w         & a         & r+        & w+        & a+        \\
                \hline
                \hline
                File must already exist     & $\bullet$ &           &           & $\bullet$ &           &           \\
                Previous contents discarded &           & $\bullet$ &           &           & $\bullet$ &           \\
                \hline
                Readable                    & $\bullet$ &           &           & $\bullet$ & $\bullet$ & $\bullet$ \\
                Writable                    &           & $\bullet$ & $\bullet$ & $\bullet$ & $\bullet$ & $\bullet$ \\
                Readable at end of file     &           &           & $\bullet$ &           &           & $\bullet$ \\
                \hline
            \end{tabular}
            \begin{tabular}{|c|c|c|}
                \hline
                \textit{type}               & Description                        & open(2) Flags                         \\
                \hline
                \hline
                \textbf{r} or \textbf{rb}   & Reading                            & \lstinline|O_RDONLY|                  \\
                \textbf{w} or \textbf{wb}   & Writing                            & \lstinline!O_WRONLY|O_CREAT|O_TRUNC!  \\
                \textbf{a} or \textbf{ab}   & Create for append                  & \lstinline!O_WRONLY|O_CREAT|O_APPEND! \\
                \textbf{r+} or \textbf{rb+} & Update (reading and writing)       & \lstinline!O_RDWR!                    \\
                \textbf{w+} or \textbf{wb+} & Create/Erase and append            & \lstinline!O_RDWR|O_CREAT|O_TRUNC!    \\
                \textbf{a+} or \textbf{ab+} & Open/Create for reading and append & \lstinline!O_RDWR|O_CREAT|O_APPEND!   \\
                \hline
            \end{tabular}
            \subsection*{Buffering}
            \begin{itemize}
                \item The goal of buffering is to use minimum \# of read/write calls.
                \item Three types:
                      \begin{itemize}
                          \item Fully Buffered: I/O performed when buffer is full
                          \item Line Buffered: I/O performed when newline is encountered
                          \item Unbuffered: I/O performed immediately
                      \end{itemize}
            \end{itemize}
            \subsubsection*{setbuf(): turns buffering on/off}
            \begin{lstlisting}
                void setbuf(FILE *fp, char *buf);
                - a buffer of length BUFSIZ (stdio.h)
                - NULL (disable buffering)
            \end{lstlisting}
            \subsubsection*{fclose: close an open stream}
            \begin{lstlisting}
                int fclose(FILE *fp);
                - Output buffer data is flushed
                - Input buffer data is discarded
                - Automatically allocated buffer is released
                - returns 0 if OK, EOF on error
            \end{lstlisting}
            \vspace*{-1em}
            \subsection*{3 types of Unformatted I/O}
            \begin{description}
                \item[Character-at-a-time]: getc, fgetc, getchar
                \item[Line-at-a-time]: fgets, fputs
                \item[Direct]: fread, fwrite
            \end{description}
            \subsection*{Methods}
            \textbf{Bisection Method}
            \begin{enumerate}
                \item Choose \(a\), \(b\) such that \(f(a) \cdot f(b) < 0\)
                \item Compute \(c = \frac{a+b}{2}\)
                \item If \(f(c)\) and \(f(a)\) have opposite signs, then \(b = c\)
                \item If \(f(c)\) and \(f(b)\) have opposite signs, then \(a = c\)
                \item If \(|b - a| < \epsilon\) or \(f(c) = 0\) then \(c\) is the solution
            \end{enumerate}
            \begin{equation*}
                \begin{aligned}
                    \text{Error} = \frac{b - a}{2^{n + 1}}                                                         & \\
                    \text{Steps(\(\epsilon\))} = \bigl\lceil \frac{\ln(b-a) - \ln(\epsilon)}{\ln(2)}-1 \bigr\rceil &
                \end{aligned}
            \end{equation*}
            \textbf{Newton's Method}
            \begin{enumerate}
                \item \(x_{n+1} = x_n - \frac{f(x_n)}{f'(x_n)}\)
                \item If \(|x_n - x_{n-1}| < \epsilon\) or \(f(x_n) = 0\) then \(x_n\) is the solution
            \end{enumerate}
            Multiplicity
            \begin{equation*}
                \begin{aligned}
                    \text{If } f^{(m-1)}(r_0) = 0 \text{ and } f^{(m)}(r_0) \neq 0 \text{,} & \\
                    \text{then } f^{(m)}(r_0) \text{ is a root of multiplicity } m          &
                \end{aligned}
            \end{equation*}
            Convergence
            \vspace*{-1em}
            \setlength{\columnsep}{-1in}
            \begin{multicols}{2}
                \begin{minipage}{1in}
                    \(m = 1\)
                    \begin{equation*}
                        \begin{aligned}
                             & M = \frac{1}{2} \left| \frac{f''(r_0)}{f'(r_0)} \right| \\
                             & e_{i+1} \approx Me_i^2                                  \\
                        \end{aligned}
                    \end{equation*}
                \end{minipage} \\
                \begin{minipage}{1in}
                    \(m > 1\)
                    \begin{equation*}
                        \begin{aligned}
                            e_{i+1} \approx \frac{m-1}{m} e_i
                        \end{aligned}
                    \end{equation*}
                \end{minipage}
            \end{multicols}
            \textbf{Secant Method}
            \begin{equation*}
                x_{n+1} = x_n - f(x_n)\frac{x_n - x_{n-1}}{f(x_n) - f(x_{n-1})}
            \end{equation*}
            Convergence
            \begin{equation*}
                \begin{aligned}
                    e_{i+1} \approx Me_i^p                            & \\
                    p = \frac{1 + \sqrt{5}}{2} \text{ (Golden Ratio)} &
                \end{aligned}
            \end{equation*}
            \textbf{False Position Method} \\
            Same as Bisection Method, but
            \begin{equation*}
                c = a - \frac{f(a) (b-a)}{f(b) - f(a)}
            \end{equation*}
            Convergence - Generally same as Bisection \\
            \textbf{Fixed Point Iteration}
            \begin{enumerate}
                \item Solve \(f(x)\) for \(x = g(x)\) \\
                      \begin{align*}
                          g(x) = 2.8x - x^2 = x \\
                          1.8x -x^2 = 0         \\
                          x(1.8 - x) = 0        \\
                          x = 0, 1.8
                      \end{align*}
                      \begin{align*}
                          f(x) = 2x^3 - 6x - 1 \\
                          g(x) = \frac{2x^3-1}{6}
                      \end{align*}
                \item If \(|g'(r_n)| < 1\), fixed point will converge to \(r_n\)
            \end{enumerate}
        \end{flushright}
    \end{minipage}
\end{multicols}
\end{document}