\documentclass[10pt]{article}

\usepackage[tmargin=0.1in, bmargin=0.25in, lmargin = 0.1in, rmargin = 0.25in]{geometry}
\usepackage{amsmath, amssymb, amsthm}
\usepackage{enumitem}
\usepackage{multicol}
\usepackage{listings}
\usepackage{pstricks}
\usepackage{tikz}
\usepackage{makecell}
\usepackage{caption}
\newcommand*\circled[1]{\tikz[baseline=(char.base)]{
    \node[shape=circle,draw,inner sep=0.5pt] (char) {#1};}}

\makeatletter
\renewcommand*\env@matrix[1][*\c@MaxMatrixCols c]{%
    \hskip -\arraycolsep
    \let\@ifnextchar\new@ifnextchar
    \array{#1}}
\makeatother

\begin{document}
\scriptsize
\begin{multicols}{2}
    \begin{minipage}{\columnwidth}
        \subsection*{Password File}
        \begin{itemize}[leftmargin=*]
            \setlength{\itemsep}{0em}
            \item ASCII text: /etc/passwd
            \item Passed structure in <pwd.h>
            \item \textbf{root}: superuser, UID=0\\root:x:0:0:root:/root:/bin/bash
            \item \textbf{guest}: no privileges\\nobody:x:65534:35534:Nobody:/home/bin/sh
            \item normal account\\elglaly:x:115:125:Yasmine Elg:/home/elglaly:/bin/bash
            \item Encrypted passwords /etc/shadow or /etc/master.passwd
            \item Shadow is readable only by root
            \item /etc/passwd is world-readable
            \item Cannot access encrypted passwords!
        \end{itemize}
        \subsection*{Group File}
        \begin{multicols*}{2}
            \begin{minipage}{0.5\columnwidth}
                \begin{itemize}[leftmargin=*]
                    \setlength{\itemsep}{0em}
                    \item /etc/group
                          \begin{itemize}
                              \item wheel (BSD)
                              \item gid 0
                          \end{itemize}
                \end{itemize}
            \end{minipage}
            \columnbreak
            \begin{minipage}{0.3\columnwidth}
                \begin{tabular}{|c|c|}
                    \hline
                    Description        & \makecell{struct group      \\member} \\
                    \hline \hline
                    group name         & \lstinline|char *gr_name|   \\
                    encrypted password & \lstinline|char *gr_passwd| \\
                    group ID           & \lstinline|int gr_gid|      \\
                    array to names     & \lstinline|char **gr_mem|   \\
                    \hline
                \end{tabular}
            \end{minipage}
        \end{multicols*}
        % \vspace*{1.5em}
        % \begin{tabular}{|c|c|cccc|}
        %     \hline
        %     Description        & POSIX.1    & \makecell{FreeBSD                                        \\ 8.0} & \makecell{Linux\\3.2.0} & \makecell{Mac OS X\\10.6.8} & \makecell{Solaris\\10} \\
        %     \hline \hline
        %     group name         & $\bullet $ & $\bullet $        & $\bullet $ & $\bullet $ & $\bullet $ \\
        %     encrypted password &            & $\bullet $        & $\bullet $ & $\bullet $ & $\bullet $ \\
        %     group ID           & $\bullet $ & $\bullet $        & $\bullet $ & $\bullet $ & $\bullet $ \\
        %     array to names     & $\bullet $ & $\bullet $        & $\bullet $ & $\bullet $ & $\bullet $ \\
        %     \hline
        % \end{tabular}
        \subsection*{Group Info}
        \begin{lstlisting}
#include <grp.h>
struct group *getgrgid(gid_t gid);
struct group *getgrnam(const char *name);
       Return: pointer if OK, NULL on error
struct group *getgrent(void); // read next entry
       Returns: pointer if OK, NULL on error/EOF
void setgrent(void); //rewind group file
void endgrent(void); // close group file
        \end{lstlisting}
        % struct group {
        %     char *gr_name; /* group name */
        %     char *gr_passwd; /* group password */
        %     gid_t gr_gid; /* group id */
        %     char **gr_mem; /* group members */
        % };
        \textbf{Supplementary}
        \begin{lstlisting}
#include <unistd.h>
int getgroups(int gidsetsize, gid_t grouplist[]);
    //Fills grouplist with gidsetsize group IDs
       Returns: #supp GIDs if OK, -1 on error

#include <grp.h> //on Linux
#include <unistd.h> //on FreeBSD, Mac OS X, Solaris
int setgroups(int ngroups, const gid_t grouplist[]);
    //set supplementary GID for calling process
int initgroups(const char *username, gid_t basegid);
//reads entire group file (getgrent, setgrent, endgrent)
       Both return: 0 if OK, -1 on error 
        \end{lstlisting}
        \begin{lstlisting}
struct passwd { /* Linux version */
    char *pw_name;  /* username */
    char *pw_passwd; /* encrypted password */
    uid_t pw_uid;   /* user ID */
    gid_t pw_gid;   /* group ID */
    char *pw_gecos; /* general info */
    char *pw_dir;   /* home directory */
    char *pw_shell; /* shell program */
};
        \end{lstlisting}
        \vspace*{-2em}
        \subsubsection*{Memory Allocation - Space Calculations}
        \textbf{sizeof} for basic types \\
        \begin{tabular}{ll}
            \hline
            \lstinline|sizeof(char)|    & = 1     \\
            \lstinline|sizeof(short)|   & = 2     \\
            \lstinline|sizeof(int)|     & = 4     \\
            \lstinline|sizeof(float)|   & = 4     \\
            \lstinline|sizeof(long)|    & = 8     \\
            \lstinline|sizeof(double)|  & = 8     \\
            \lstinline|sizeof(char *) | & = 4 / 8 \\
            \hline
        \end{tabular}
        \\
        \textbf{sizeof} for array types
        \begin{lstlisting}
double sample[100];
sizeof(sample) = 100 * 8 = 800
char string[81];
sizeof(string) = 81 * 1 = 81
        \end{lstlisting}
        BUT
        \begin{lstlisting}
void foo(char buffer[81]) { . . . }
sizeof(buffer); // = 4 or 8 !!
\end{lstlisting}
        Array arguments are really pointers!
        \tiny
        \begin{lstlisting}
struct tm {
    int tm_sec; /* Seconds (0-60) */
    int tm_min; /* Minutes (0-59) */
    int tm_hour; /* Hours (0-23) */
    int tm_mday; /* Day of the month (1-31) */
    int tm_mon; /* Month (0-11) */
    int tm_year; /* Year - 1900 */
    int tm_wday; /* Day of the week (0-6, Sunday = 0) */
    int tm_yday; /* Day in the year (0-365, 1 Jan = 0) */
    int tm_isdst; /* Daylight saving time */
};
\end{lstlisting}
    \end{minipage}
    \begin{minipage}{1.5\columnwidth}
        \begin{multicols*}{2}
            \hspace*{-9.5em}
            \begin{minipage}{0.8\columnwidth}
                \vspace*{-15em}
                \textbf{Time [ctd.]}
                \begin{lstlisting}
char *asctime(const struct tm *tm);
 Put date & time in standard format
 Ex: fputs(asctime(localtime), stdout)
 -> Wed Oct 21 13:02:36 2020
strftime(3); // format options
 e.g. full day name, replace day of
 month by decimal
ctime(3); // adjusts the time value for
 the current time zone
struct tm *localtime(const time_t *timep)
 convert time to local time
                \end{lstlisting}
            \end{minipage}
            \columnbreak
            \begin{minipage}{0.5\columnwidth}
                \captionof*{table}{Fields in \lstinline|/etc/shadow|}
                \hspace*{3em}
                \begin{tabular}{|c|c|c|}
                    \hline
                    Description             & \makecell{\lstinline|struct passwd| \\member} \\
                    \hline \hline
                    username                & \lstinline|char *pw_name|           \\
                    encrypted password      & \lstinline|char *pw_passwd|         \\
                    user ID                 & \lstinline|uid_t pw_uid|            \\
                    group ID                & \lstinline|gid_t pw_gid|            \\
                    comment                 & \lstinline|char *pw_gecos|          \\
                    workding directory      & \lstinline|char *pw_dir|            \\
                    shell program           & \lstinline|char *pw_shell|          \\
                    access class            & \lstinline|char *pw_class|          \\
                    time to change passwd   & \lstinline|time_t pw_change|        \\
                    account expiration date & \lstinline|time_t pw_expire|        \\
                    \hline
                \end{tabular}
                \captionof*{table}{Fields in \lstinline|/etc/passwd|}
                \begin{tabular}{|c|c|c|}
                    \hline
                    Description                  & \makecell{\lstinline|struct spwd| \\member} \\
                    \hline \hline
                    user login name              & \lstinline|char *sp_namp|         \\
                    encrypted password           & \lstinline|char *sp_pwdp|         \\
                    date of last change          & \lstinline|int sp_lstchg|         \\
                    days until change allowed    & \lstinline|int sp_min|            \\
                    days until change required   & \lstinline|int sp_max|            \\
                    days before warning          & \lstinline|int sp_warn|           \\
                    days before account inactive & \lstinline|int sp_inact|          \\
                    date when account expires    & \lstinline|int sp_expire|         \\
                    reserved                     & \lstinline|unsigned int sp_flag|  \\
                    \hline
                \end{tabular}
            \end{minipage}
        \end{multicols*}
    \end{minipage}
    \begin{minipage}{\columnwidth}
        \begin{flushright}
            \vspace*{2em}
            \subsection*{Program Access}
            \textbf{Fetching Entries}
            \begin{lstlisting}
    #include <pwd.h>
    struct passwd *getpwuid(uid_t uid); // used by ls
    struct passwd *getpwnam(const char *name);
                // used by login
            Return: pointer to passwd struct if OK,
                    NULL on error
            \end{lstlisting}
            \textbf{Iteration}
            \begin{lstlisting}
    struct passwd *getpwent(void);
                // Opens necessary files
            Return: pointer to passwd struct if OK,
                    NULL on error/EOF
    void setpwent(void); // rewinds files
    void endpwent(void); // closes files
            \end{lstlisting}
            \textbf{Shadow Passwords}
            \begin{lstlisting}
    #include <shadow.h>
    struct spwd *getspnam(const char *name);
    struct spwd *getspent(void);
            Return: pointer if OK, NULL on error
    void setspent(void);
    void endspent(void);
            \end{lstlisting}
            \vspace*{-3em}
            \subsection*{Dynamic Memory Allocation}
            \begin{lstlisting}
    #include <stdlib.h>
    void *malloc(size_t size);
    void *calloc(size_t nobj, size_t size);
    void *realloc(void *ptr, size_t newsize);
        Returns: pointer on success, NULL otherwise
    void free(void *ptr);
            \end{lstlisting}
            \lstinline|void *malloc( unsigned nbytes )|
            \begin{itemize}
                \setlength{\itemsep}{-0.5em}
                \item Allocates \lstinline|nbytes| of memory
                \item Guaranted not to overlap other allocated memory
                \item Returns point to first byte (or NULL if heap is full)
                \item Similar to constructor in Java - allocates space
                \item Allocated space is uninitialized (random garbage)
            \end{itemize}
            \lstinline|void free( void *ptr )|
            \begin{itemize}
                \setlength{\itemsep}{-0.5em}
                \item Frees the memory assigned to ptr.
                \item The space must have been allocated by \lstinline|malloc|
                \item No garbage collection in C
                \item Can slowly consume memory if not careful
            \end{itemize}
            \vspace*{-4em}
            \subsection*{Time}
            \begin{lstlisting}
#include <time.h>
time_t time(time_t *calptr);
       Returns: value of time if OK, -1 on error
Number of seconds since Epoch: 00:00:00 1970/1/1, UTC
Example: curtime = time (NULL); /* Get the current time. */

#include <sys/time.h>
int gettimeofday(struct timeval *restrict tp,
    void *restrict tzp);
       Returns: 0 always
struct timeval {
    time_t tv_sec; /* sec */
    long tv_usec; /*microsec*/
};
    \end{lstlisting}
        \end{flushright}
    \end{minipage}
\end{multicols}
\end{document}
