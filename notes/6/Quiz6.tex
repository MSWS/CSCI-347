\documentclass[10pt]{article}

\usepackage[tmargin=0.1in, bmargin=0.25in, lmargin = 0.1in, rmargin = 0.25in]{geometry}
\usepackage{amsmath, amssymb, amsthm}
\usepackage{enumitem}
\usepackage{multicol}
\usepackage{listings}
\usepackage{pstricks}
\usepackage{makecell}
\usepackage{caption}
\newcommand*\circled[1]{\tikz[baseline=(char.base)]{
    \node[shape=circle,draw,inner sep=0.5pt] (char) {#1};}}

\makeatletter
\renewcommand*\env@matrix[1][*\c@MaxMatrixCols c]{%
    \hskip -\arraycolsep
    \let\@ifnextchar\new@ifnextchar
    \array{#1}}
\makeatother

\begin{document}
\footnotesize
\begin{multicols}{2}
    \begin{minipage}{\columnwidth}
        \normalsize
        \subsection*{Parent and Child}
        If the call to \textbf{fork()} is executed successfully:
        \begin{itemize}
            \item two identical copies of address space created
            \item one for parent, one for child
            \item both programs execute independently
        \end{itemize}
        \subsubsection*{Exit Functions}
        Termination Status:
        \begin{itemize}
            \item normal: exit status
            \item abnormal: kernel indicates reason
        \end{itemize}
        Child returns termination status to parent \\
        (using wait or waitpid)
        \subsubsection*{Termination Conditions}
        A terminated process whose parent has not \\
        waited for is a \textit{zombie}.
        \begin{itemize}
            \item parent can't check child's status
            \item Kernel keeps minimal info of child process \\
                  (pid, status, CPU time)
        \end{itemize}
        If parent terminates before child:
        \begin{itemize}
            \item Kernel assigns init (pid=1) to be parent of child
            \item init's inherited child do not become zombies\\
                  (wait() fetches status)
        \end{itemize}
        \subsubsection*{Zombies}
        \begin{itemize}
            \item A zombie does not use a lot of memory
            \item The problem occurs when you have a lot of zombies
            \item Limited PIDs
        \end{itemize}

        \subsection*{System Function}
        Implemented by calling fork, exec, and waitpid.
        \begin{lstlisting}
#include <stdlib.h>
int system(const char *cmdstring);
  Returns:
    -1 with errno if fork or waitpid fails
    127 as if shell exit(127) if
      shell cannot execute command
    Termination status otherwise
    \end{lstlisting}

        \subsection*{Interpreter Files (Shebang)}
        Script files that begin with:
        \begin{itemize}
            \item \texttt{\#!pathname [optional-argument]}
            \item Eg: \texttt{\#!/bin/bash}, \texttt{\#!/bin/csh}
        \end{itemize}
        Allows users an easy and efficient way to execute some commands using
        scripts.
        Ensure file is executable: \texttt{chmod +x filename} \\
        Execing a script file:
        \texttt{execl("/bin/testinterp", "testinterp", "myarg1", NULL);}
    \end{minipage}
    \begin{minipage}{\columnwidth}
        \subsection*{Threads}
        \textbf{One process can have multithreads}
        \begin{itemize}
            \item Each thread handles a separate task
            \item Threads have access to same memory address and file descriptors
            \item Multithreaded process can run on a uniprocessor
            \item Ex for word processor:
                  \begin{itemize}
                      \item Background thread checks spelling / grammar
                      \item Foreground thread handles user input
                      \item Third thread loads images from hard drive
                      \item Fourth thread does automatic saves
                  \end{itemize}
        \end{itemize}
        \subsubsection*{Identification}
        A thread's ID is represented by \texttt{pthread\_t} type.
        The \texttt{pthread\_equal} function is used to compare two IDs.

        \begin{lstlisting}
  #include <pthread.h>
  int pthread_equal(pthread_t tid1, pthread_t tid2);
      Returns:
        nonzero if equal, 0 otherwise
  pthread_t pthread_self(void);
      Returns
        thread ID of calling thread
          \end{lstlisting}
        \subsubsection*{Creation (pthread\_create(3))}
        \begin{lstlisting}
  #include <pthread.h>
  int pthread_create(pthread_t *thread,
      const pthread_attr_t *attr,
      void *(*start_rtn)(void *), void *arg);
        \end{lstlisting}
        \begin{itemize}
            \setlength{\itemsep}{0em}
            \item \texttt{tidp}: new thread id
            \item \texttt{attr}: thread attributes
            \item \texttt{start\_rtn}: function to be executed
            \item \texttt{arg}: argument to be passed to \texttt{start\_rtn}
        \end{itemize}
        \subsubsection*{Termination}
        If any thread in a process calls \texttt{exit}, \texttt{\_exit}, or
        \texttt{\_Exit}, the entire process terminates.
        When default action is to terminate the process, a \texttt{signal}
        sent to a thread will terminate the entire process.
        A single thread can exit in three ways without affecting the entire process:
        \begin{enumerate}
            \item Thread can \texttt{return} from start routine, returned value is
                  thread's exit status
            \item Thread can be \texttt{canceled} by another thread in same process
            \item Thread can call \texttt{pthread\_exit}
        \end{enumerate}
        \begin{lstlisting}
  #include <pthread.h>
  void pthread_exit(void *rval_ptr);
  int pthread_join(pthread_t thread, void **rval_ptr);
      /* similar to wait */
      Returns: 0 if successful, error number otherwise
  int pthread_cancel(pthread_t thread);
      /* like pthread_exit with arg of PTHREAD_CANCELED */
      Returns: 0 if successful, error number otherwise
        \end{lstlisting}
        \hspace*{-3em}
        \begin{tabular}{|c|c|c|}
            \hline
            Process primitive & Thread primitive                & Description                            \\
            \hline \hline
            \texttt{fork}     & \texttt{pthread\_create}        & Create new flow of control             \\
            \texttt{exit}     & \texttt{pthread\_exit}          & Exit from existing flow of control     \\
            \texttt{waitpid}  & \texttt{pthread\_join}          & Wait for flow of control to terminate  \\
            \texttt{atexit}   & \texttt{pthread\_cleanup\_push} & Register function to be called at exit \\
            \texttt{getpid}   & \texttt{pthread\_self}          & Get ID of flow of control              \\
            \texttt{abort}    & \texttt{pthread\_cancel}        & Terminate flow of control              \\
            \hline
        \end{tabular}

        \subsubsection*{Cleanup}
        \begin{lstlisting}
  #include <pthread.h>
  void pthread_cleanup_push(void (*rtn)(void *), void *arg);
    /* Called when thread exits */
  void pthread_cleanup_pop(int execute);
    /* Removes cleanup handler establish by last
       call to pthread_cleanup_push */
  int pthread_detach(pthread_t thread);
        \end{lstlisting}
    \end{minipage}
\end{multicols}
\end{document}
