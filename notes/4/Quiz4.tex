\documentclass[10pt]{article}

\usepackage[tmargin=0.1in, bmargin=0.25in, lmargin = 0.1in, rmargin = 0.25in]{geometry}
\usepackage{amsmath, amssymb, amsthm}
\usepackage{enumitem}
\usepackage{multicol}
\usepackage{listings}
\usepackage{pstricks}
\usepackage{makecell}
\usepackage{caption}
\newcommand*\circled[1]{\tikz[baseline=(char.base)]{
    \node[shape=circle,draw,inner sep=0.5pt] (char) {#1};}}

\makeatletter
\renewcommand*\env@matrix[1][*\c@MaxMatrixCols c]{%
    \hskip -\arraycolsep
    \let\@ifnextchar\new@ifnextchar
    \array{#1}}
\makeatother

\begin{document}
\scriptsize
\begin{multicols}{2}
    \begin{minipage}{\columnwidth}
        \subsection*{Process Lifecycle}
        \begin{itemize}
            \setlength{\itemsep}{-0em}
            \item int main(void) \{ /* .. */ \}
            \item int main(int argc, char* argv[]) \{ /* ... */ \}
            \item int main(int argc, char* argv[], char* envp[]) \{ /* ... */ \}
        \end{itemize}
        \begin{itemize}
            \setlength{\itemsep}{-0em}
            \item When exec func is called, kernel needs to start the given program
            \item Special startup routine is called by kernel which sets up main
            \item C startup routine sets up environment and args into proper registers
            \item main returns an int which is passed to exit(3)
            \item When a program is started, the call could be exit(main(argc, argv))
        \end{itemize}
        \subsubsection*{Termination}
        Normal termination:
        \begin{itemize}
            \setlength{\itemsep}{0em}
            \item return from mainexit(3), \_exit(2), or \_Exit(2)
            \item return of last thread from start routine
            \item calling pthread\_exit(3) from last thread
        \end{itemize}
        Abnormal termination:
        \begin{itemize}
            \setlength{\itemsep}{0em}
            \item calling abort(3)
            \item termination by signal
            \item response of last thread to a cancellation request
        \end{itemize}
        \subsection*{Environment}
        Env vars are stored in global NULL terminated array of pointers:
        \lstinline|extern char **environ;| A similar array can be passed
        to main: \lstinline|int main(int argc, char **argv, char **envp)|
        \begin{lstlisting}
#include <stdlib.h>
char *getenv(const char *name);
    Returns: value if found, NULL if not found

int putenv(char *string);
int setenv(const char *name, const char *value,
    int overwrite);
int unsetenv(const char *name);
    Returns: 0 if OK, -1 on error
        \end{lstlisting}
        \subsubsection*{wait(2) and waitpid(2)}
        \begin{lstlisting}
#include <sys/wait.h>
pid_t wait(int *status);
pid_t waitpid(pid_t pid, int *status, int options);

#include <sys/resource.h>
pid_t wait3(int *status, int options, struct rusage *rusage);
pid_t wait4(pid_t pid, int *status, int options,
    struct rusage *rusage);
    Returns: child PID if OK, -1 on error
        \end{lstlisting}
        \begin{itemize}
            \setlength{\itemsep}{0em}
            \item wait() suspends execution of the process until status info is available for a terminated child
            \item waitpid() and wait(4) allow waiting for a specific process
            \item wait3() and wait4() allow inspection of resource usage
        \end{itemize}
        \subsubsection*{exec(3)}
        The exec() family of functions replaces the current process image with a new process image.
        They are all front-ends to the execve(2) system call.
        \begin{lstlisting}
#include <unistd.h>
int execl(const char *path, const char *arg, ...);
int execlp(const char *file, const char *arg, ...);
int execlpe(const char *file, const char *arg, ...,
    char *const envp[]);
int execle(const char *path, const char *arg, ...,
    char *const envp[]);
int execv(const char *path, char *const argv[]);
int execve(const char *path, char *const argv[],
    char *const envp[]);
int execvp(const char *file, char *const argv[]);
int execvpe(const char *file, char *const argv[],
    char *const envp[]);
    Returns: -1 on error, no return on success
        \end{lstlisting}
        \hspace*{-1em}
        \begin{tabular}{|c|c|c|c||c|c|c|c|}
            \hline
            Function & \textit{pathname} & \textit{filename} & \textit{fd} & Arg list  & \textit{argv[]} & \textbf{environ} & \textit{envp[]} \\
            \hline \hline
            execl    & $\bullet$         &                   &             & $\bullet$ &                 & $\bullet$        &                 \\
            execlp   &                   & $\bullet$         &             & $\bullet$ &                 & $\bullet$        &                 \\
            execle   & $\bullet$         &                   &             & $\bullet$ &                 &                  & $\bullet$       \\
            execv    & $\bullet$         &                   &             &           & $\bullet$       & $\bullet$        &                 \\
            execvp   &                   & $\bullet$         &             &           & $\bullet$       & $\bullet$        &                 \\
            execve   & $\bullet$         &                   &             &           & $\bullet$       &                  & $\bullet$       \\
            fexecve  &                   &                   & $\bullet$   &           & $\bullet$       &                  & $\bullet$       \\
            \hline
        \end{tabular}
    \end{minipage}
    \begin{minipage}{\columnwidth}
        \subsection*{Exits}
        \subsubsection*{exit(3)}
        \begin{lstlisting}
#include <stdlib.h>
void exit(int status);
        \end{lstlisting}
        \begin{itemize}
            \setlength{\itemsep}{0em}
            \item terminates a process, before termination:
                  \begin{itemize}
                      \setlength{\itemsep}{-0em}
                      \item Calls the functions registered with atexit(3) in reverse order
                      \item Flush all open output streams, close al open streams
                      \item Unlink all files created with tmpfile(3) function
                  \end{itemize}
            \item calls \_exit(2)
        \end{itemize}
        \subsubsection*{\_exit(2)}
        \begin{lstlisting}
#include <unistd.h>
void _exit(int status);
        \end{lstlisting}
        \begin{itemize}
            \setlength{\itemsep}{-0em}
            \item terminates a process immediately
            \item does not call functions registered with atexit(3)
        \end{itemize}
        \subsection*{Process Control}
        Process ID is a nonnegative int. IDs are guaranteed to be unique
        and identify a particlar exisintg process.
        \begin{lstlisting}
    #include <unistd.h>
    pid_t getpid(void);  // return PID
    pid_t getppid(void); // return parent PID
    uid_t getuid(void);  // return real user ID
    uid_t geteuid(void); // return effective user ID
    gid_t getgid(void);  // return real group ID
    gid_t getegid(void); // return effective group ID
        \end{lstlisting}
        \begin{tabular}{ll}
            PID     & Process                            \\
            0       & swapper (scheduler)                \\
            1       & init (/sbin/init)                  \\
            2       & pagedaemon (virtual memory paging) \\
            3,4,... & Other processes                    \\
        \end{tabular}
        \subsubsection*{fork(2)}
        \begin{lstlisting}
    #include <unistd.h>
    pid_t fork(void);
        Returns: 0 in child, PID of child in parent, -1 on error
        \end{lstlisting}
        \begin{itemize}
            \setlength{\itemsep}{0em}
            \item fork(2) is the ONLY way to create a process in Unix kernel by user
            \item Child process has unique PID
            \item copy-on-write (COW):
                  \begin{itemize}
                      \item Memory regions are read-only and shared by parent and child
                      \item when write occurs, kernel makes a copy of that memory for that process
                  \end{itemize}
            \item No guarantee of order of execution
        \end{itemize}
        \begin{minipage}{0.4\columnwidth}
            Possibe reasons for fail:
            \begin{itemize}
                \item too many processes
                \item total \# exceeds user limit
            \end{itemize}
        \end{minipage}
        \begin{minipage}{0.5\columnwidth}
            Two common uses:
            \begin{itemize}
                \item Duplicate to execute different code sections (networking)
                \item Execute different programs (shells)
            \end{itemize}
        \end{minipage}
        \begin{itemize}
            \setlength{\itemsep}{0em}
            \item Parent and child share same file descriptors
            \item As if dup() had been called on all file descriptors
            \item Parent and child share same file offset
            \item Intermixed output from parent and child
        \end{itemize}
        \begin{itemize}
            \item All processes not explicitly isntantiated by kernel were created by fork(2)
            \item fork(2) creates copy of current process including file descriptors and output buffers
            \item To replace current process with new process image, use exec(3) family of function
            \item After creating new process via fork(2), the parent process can wait(2) for child to process to reap its exist status and resource utilization
            \item Failure to wait(2) will create a zombie process until parent is terminated
        \end{itemize}
    \end{minipage}
\end{multicols}
\end{document}
