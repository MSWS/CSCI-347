\documentclass{article}

\usepackage{pgfplots}
\usepackage[utf8]{inputenc}
\usepackage{csvsimple}
\usepackage{tikz}
\usepackage{enumitem}
\usepackage{amsmath, amssymb}
\usepackage{hyperref}

\title{Edge Detector Concurrency Report}
\author{Isaac Boaz}
\pgfplotsset{compat=1.18} % Adjust compatibility if needed

\begin{document}
\maketitle

\begin{abstract}
    The goal of the edge detector project was to implement a concurrent edge
    detector program utilizing a laplacian filter.
\end{abstract}

\section{Structure}

The structure of the program follows basic C programming requirements.
Each function was written to complete its assigned task with no side effects.
The basic workflow of the program is as follows:
\begin{enumerate}
    \item Call the program with the PPM file paths to be processed.
    \item Create a \texttt{file\_name\_args} struct array for each file argument.
    \item Populate the array with appropriate file information.
    \item Create a thread for each file argument.
          \begin{enumerate}[label=4.\arabic*.]
              \item Each thread reads in the image.
              \item Creates \texttt{LAPLACIAN\_THREADS} threads to process the image.
              \item Feeds the image to the laplacian filter threads with the appropriate
                    row range.
          \end{enumerate}
    \item Waits for all threads to finish.
\end{enumerate}

\subsection{Global Variables}

The one global variable used in this program is \texttt{total\_elapsed\_time},
which is responsible for tracking the total time taken to process all images.
All other references to globals are constant defines.

\subsection{Functions}
\subsubsection{compute\_laplacian\_threadfn}
This function is responsible for doing the actual laplacian filter computation.
Given a \texttt{parameter struct} pointer, it iterates from the given \texttt{start}
through to the \texttt{end} and applies the laplacian filter to each pixel.

As the image is passed by reference, the function does not return anything.

\subsubsection{apply\_filters}
This function is responsible for creating the threads that will apply the laplacian
filter. It takes in pointers to an image, its dimensions, and a pointer to
the \texttt{elapsedTime} double.

The function creates \texttt{LAPLACIAN\_THREADS} threads and assigns each thread
a range of rows to process. The function then waits for all threads to finish.
Lastly, the function calculates the total time taken to process the image and
updates the \texttt{elapsedTime} pointer respectively.

\subsubsection{write\_image}
This function is responsible for writing a given \texttt{PPMPixel struct} given
a filename, widht, and height. A hardcoded \texttt{"P6"} is written to the file,
as well as the provided width, height, and maximum color value.

\subsubsection{read\_image}
Much like \texttt{write\_image}, this function is responsible for reading in
a \texttt{PPMPixel struct} from a given filename. The function reads in the
file header and then reads in the pixel data.

If the file does not contain the valid \texttt{"P6"} header, the function
exits early.

\subsubsection{manage\_image\_file}
This function is a basic wrapper around \texttt{read\_image},
\texttt{apply\_filters}, \\
and \texttt{write\_image}. It takes in a \texttt{file\_name\_args struct}
pointer and calls the aforementioned functions in order.

\subsubsection{main}
This function is responsible for parsing the command line arguments and
creating the necessary threads to process each image.

\section{Concurrency}
Concurrency is an important aspect of this program and how it achieves
efficency. Multhreading is heavily used to allow for processing multiple images
at multiple times.

\subsection{Threads}
As described in the above functions and structure, one thread is used for each
image file provided, with that thread creating subsequent threads to handle each
image.

As a result, the number of threads created can be calculated as follows:
\begin{equation*}
    \text{num\_threads} = \text{num\_files} + \text{num\_files} \times \text{LAPLACIAN\_THREADS}
\end{equation*}

An analysis of how the value of \texttt{LAPLACIAN\_THREADS} affects the program
is provded in \autoref{sec:Results}.

\subsection{Race Conditions}
Race conditions were avoided by ensuring that each thread was given a unique
range of rows to process. This was done by calculating the number of rows
to process and then dividing that number by the number of threads to create
per image.

Since there are no global variables used for storing arrays or data, (all
image is passed by reference), there is no possibility of a race condition
occurring.

\section{Experiments}
The experiment run was executed on Western Washington University's Lab Machines. Specifically,
the \texttt{cf167-23} was used for this experiment. All tests used the same base image (cayuga\_1.ppm)
which was 1440060 bytes in size.

\subsection{Conditions}
The primary question to be answered by the experiment was how the number of threads used
to process the laplacian filter affected the total time taken to process the image.

The number of threads used to process the laplacian filter was varied from 1 up to 1028 threads.
With each variance, the edge detector was ran 100 times and the average time taken to process the image
was recorded.

\section{Results}
\label{sec:Results}

The results of the experiment are graphed in \autoref{fig:Results}.

\begin{figure}
    \centering
    \begin{tikzpicture}
        \begin{axis}[
                xlabel={Threads},
                ylabel={Seconds},
                title={Seconds vs. Threads}
            ]
            \addplot+ table [x=Threads, y=Total, col sep=comma] {data.csv};
        \end{axis}
    \end{tikzpicture}
    \caption{Seconds vs. Threads}
    \label{fig:Results}
\end{figure}

\section{Results Analysis}
Based on the results, it is clear that the number of threads used to process the laplacian filter
has a significant impact on the total time taken to process the image. As the number of threads
increases from 1 to 64, the total time taken to process the image decreases. However, after 64 threads,
the total time taken to process the image increases as the number of threads increases.

It should be noted that the results of this experiment may vary depending on the image used.
Outside of software differences, the hardware used to run the experiment may also affect the results.

\section{Conclusion}
This program was successfully able to utilize an arbitrary amount of threads (greater than zero) to
appropriately process an image using a laplacian filter. The results show that the performance of
program is signficatnly altered off of the number of threads used.

It is likely that the number of threads used could be optimized based off the image size. Further
experiments could be run to determine the optimal number of threads to use for a given image size.

Additionally, experiments for how the program handles multiple images at once
(instead of one at a time many times) could be run to determine how the program
scales with multiple images.

\end{document}